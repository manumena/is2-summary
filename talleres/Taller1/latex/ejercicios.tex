\section*{Ejercicio 1}
\setcounter{section}{1}

Se corrió SOOT en MacOS para el programa de el ejercicio 1 mediante el
siguiente comando 
\\

java -cp .:soot-3.3.0-jar-with-dependencies.jar:. soot.Main -cp .:\$JRE -f J Exercise1 -print-tags -p jap.rdtagger enabled:true -p jb use-original-names:true -p jb.cp off -keep-line-number
\\

Además se analizaron las Reaching Definitions mediante el algoritmo de Chaotic
Iteration con el objetivo de complementar y explicar el análisis hecho por SOOT:
\\

\begin{tabular}{| c | c | c |}
\hline
$\#$		& $In$								&$Out$ \\
\hline
$0$			& $-$								&$\{<a,?>, <c,?>\}$ \\
\hline
$1$			& $\{<a,?>, <c,?>\}$				&$\{<a,1>, <c,1>\}$ \\
\hline
$2$			& $\{<a,1>, <c,1>\}$				&$\{<a,1>, <c,1>\}$ \\
\hline
$3$			& $\{<a,1>, <c,1>\}$				&$\{<a,3>, <c,1>\}$ \\
\hline
$5$			& $\{<a,1>, <c,1>, <a,3>\}$			&$\{<a,1>, <c,5>, <a,3>\}$ \\
\hline
$6$			& $\{<a,1>, <c,1>, <a,3>, <c,7>\}$	&$\{<a,1>, <c,5>, <a,3>, <c,7>\}$ \\
\hline
$7$			& $\{<a,1>, <c,5>, <a,3>, <c,7>\}$	&$\{<a,1>, <c,7>, <a,3>\}$ \\
\hline
$9$			& $\{<a,1>, <c,5>, <a,3>, <c,7>\}$	&$\{<c,5>, <c,7>, <a,9>\}$ \\
\hline
$10$		& $\{<c,5>, <c,7>, <a,9>\}$			&$\{<c,5>, <c,7>, <a,9>\}$ \\
\hline
\end{tabular}

\subsection{}
El output del comando de SOOT no nos muestra ninguna definición de variables
alcanzada por la línea $\#5$. Esto es así porque SOOT omite las definiciones que
no pueden usarse para calcular esa definición, en este caso la asignación de la
constante $1$. 

Por esto se requiere un análisis propio. Como podemos observar en la tabla las
definiciones alcanzadas hasta la linea $\#5$ (viendo el IN) son las de a\#1, c\#1
y a\#3, $a = 8$, $c = 3$ y $a = 5$, respectivamente. Esto se debe a que este
dataflow analysis tiene en cuenta todos los caminos posibles
(incluyendo los que podrian nunca ser alncanzados) por lo cual se dice que en
este análisis se sacrifica completeness para ganar soundness.

\subsection{}
El output de SOOT para la línea $\#9$ muestra lo siguiente:
\\

/*c\#3 has reaching def: c\#3 = 1*/ \\

/*c\#3 has reaching def: c\#3 = c\#3 + 2*/ \\

/*a has reaching def: a = 8*/ \\

/*a has reaching def: a = 5*/ \\
\\

lo cual se condice con el análisis expresado en la tabla. La variables $c$
alcanza las definiciones de las líneas 5 y 7, y la $a$ las de las líneas 1 y 3.
No se alcanza la definicion de $c$ de la línea $\#1$ ya que es parte del
conjunto KILL de la línea $\#5$.

\subsection{}
El output de SOOT para la línea $\#10$ muestra:
\\

/*a\#8 has reaching def: a\#8 = c\#3 - a*/ \\

Como vemos, coincide con el resultado de la tabla: la definicion de $a$ 
alcanzada es la de la línea $\#9$, ya que en ella se sobreescriben las
definiciones de $a$ de las líneas anteriores. Como en el primer inciso, SOOT no
escribe sobre las definiciones que no se utilizan dentro de la línea, en este
caso las de $c$. Si miramos la tabla observamos que la línea $\#10$ también es
alcanzada por las definiciones de $c$ de las líneas 5 y 7.

\section*{Ejercicio 2}
\setcounter{section}{2}
Se corrió SOOT en MacOS para el programa de el ejercicio 2 mediante el
siguiente comando (igual que en el ejercicio anterior pero para el archivo del
ejercicio 2, y ademas analizando Live Variables en lugar de Reaching Definitions)
\\

java -cp .:soot-3.3.0-jar-with-dependencies.jar:. soot.Main -cp .:\$JRE -f J Exercise2 -print-tags -p jap.lvtagger enabled:true -p jb use-original-names:true -p jb.cp off -keep-line-number
\\

Una vez más, se analizaron las Live Variables mediante el algoritmo de Chaotic
Iteration con el objetivo de complementar y explicar el análisis hecho por SOOT:
\\

\begin{tabular}{| c | c | c |}
\hline
$\#$		& $In$				&$Out$ \\
\hline
$0$			& $-$				&$\{a, b\}$ \\
\hline
$1$			& $\{a, b\}$		&$\{a, b, c\}$ \\
\hline
$2$			& $\{a, b, c\}$		&$\{a, b, c, d\}$ \\
\hline
$3$			& $\{a, b, c, d\}$	&$\{a, b, c, d\}$ \\
\hline
$4$			& $\{a, b, c, d\}$	&$\{c, d\}$ \\
\hline
$5$			& $\{c\}$			&$\{r\}$ \\
\hline
$7$			& $\{d\}$			&$\{r\}$ \\
\hline
$9$			& $\{r\}$			&$-$ \\
\hline
\end{tabular}
\\

El análisis de Live Variables es backward, a diferencia del del punto anterior
que es forward. Esto significa que se comienza a analisar desde el último nodo
hacia el primero. 

En este caso nos interesará observar el conjunto $OUT$ de cada línea para
determinar qué variables estan vivas.

\subsection{}
El output de SOOT para la línea $\#5$ muestra:
\\

/*Live Variable: r*/
\\

Esto coincide con el resultado que obtuvimos en la tabla mediante chaotic
iteration. Significa que en algun camino siguiente se utilizará (en la línea 
$\#9)$ esta variable antes de ser redefinida.

\subsection{}
El output de SOOT para la línea $\#7$ es el mismo que para la $\#5$:

/*Live Variable: r*/
\\

ya que en ambos casos $r$ es la única variable por ser utlizada

\subsection{}
SOOT no muestra resultados para la línea $\#9$ y es debido a que luego de esta
línea el programa finalizará por lo que no pueden haber variables vivas, es
decir, que fueran a ser utilizadas antes de redefinirse.

\section*{Ejercicio 3}
El null pointer checker analiza si en algún punto del programa un puntero apunta a \textit{NULL}.

Se corrió el análisis de SOOT y se obtuvo el siguiente output:

\begin{verbatim}
public class C extends java.lang.Object
{
/*C.java*/
/*[inner=C$Cell, outer=C, name=Cell,flags=10]*/
/*soot.tagkit.InnerClassAttribute@4653c98c*/

/*0*/
 public void <init>()
 {
     C this;

     this := @this: C;

     specialinvoke this.<java.lang.Object: void <init>()>();
/*1*/
/*[not null]*/

     return;
/*1*/
 }

/*7*/
 public int exercise3(C$Cell, C$Cell)
 {
     C this;
     C$Cell c1, c2;
     int $stack3;

     this := @this: C;

     c1 := @parameter0: C$Cell;

     c2 := @parameter1: C$Cell;

     c1.<C$Cell: int value> = 1;
/*8*/
/*[unknown]*/

     c2.<C$Cell: int value> = 2;
/*9*/
/*[unknown]*/

     $stack3 = c1.<C$Cell: int value>;
/*10*/
/*[not null]*/

     return $stack3;
/*10*/
 }
}
\end{verbatim}

En la línea 3 ninguna de las variables puede estar apuntando a \textit{NULL}. Esto es porque al pasarse como parámetros \textit{c1} y \textit{c2} se sabe que tienen un campo denominado \textit{value}, porque ambos pertenecen a la clase \textit{Cell}, que tiene siempre el campo \textit{value}.

Con lo cual, \textit{c1} y \textit{c2} son punteros a una estuctura que tiene un campo \textit{value} que puede almacenar un entero.

Antes de ejecutar la línea 3, se escriben tanto el valor de \textit{c1.value} como el de \textit{c2.value}, con los valores enteros 1 y 2, respectivamente.

Por lo tanto, al llegar a la línea 3 ninguno de los punteros en el programa puede estar apuntando a un valor nulo.